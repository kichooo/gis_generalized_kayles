\documentclass[11pt,a4paper]{article}
\usepackage{polski}
\usepackage[utf8]{inputenc}
\usepackage{listings}

\title{GIS Sprawozdanie II}
\author{Krzysztof Burliński, Piotr Kałamucki}



\begin{document}
\maketitle
\tableofcontents
\newpage
\section{Treść zadania}Dany jest graf G=(V,E). Gracze na zmianę wybierają wierzchołek w grafie, który następnie jest usuwany razem ze wszystkimi przyległymi wierzchołkami. Gracz wygrywa wtedy i tylko wtedy gdy jego przeciwnik nie może wykonać ruchu, czyli nie ma już żadnego wierzchołka do wyboru.
\section{Algorytm}
Poszukując algorytmu dla zadanego projektu, zaczęlismy od rozwiązań poprawnych. Nie mogąc takiego znaleźć zacząliśmy, tak jak jest w temacie projektu, poszukiwać odpowiedniej heurestyki dla tego problemu. Generalized Kayles jest uogólnieniem gry Kayles z tablicy na graf (z racji, że nie było to sprecyzowane w treści, przyjęliśmy skierowany). 

Strategia dla gry Kayles została już opracowana, gdyż gra ta należy do zbioru gier symetrycznych. Gracz rozpoczynający może zapewnić sobie zwycięstwo dzieląc listę na dwie części, a potem powielając ruchy przeciwnika. W ten sposób, pierwszym gracz, który nie będzie mógł wykonać ruchu staje się gracz, który wykonuje ruch jako drugi.

W przypadku Generalized Kayles sytuacja nie jest niestety bardzo podobna to oryginału. Głównym powodem jest to, że pierwszy gracz nie może po prostu (poza rzadkimi przypadkami drzew symetrycznych) dokonać podziału grafu na dwie części za pomocą jednego ruchu. 

Po analizie zasad gry udało nam się zauważyć, że gracz może być pewny zwycięstwa w chwili, gdy w grze pozostaje nieparzysta liczba grafów posiadających wierzchołek sąsiadujący z wszystkimi innymi wierzchołkami. Jest do jednak dość rzadka sytuacja, ponadto, zasady tej nie udało nam się uogólnić na grafy nie posiadające tej własności.

Kolejną dokonaną obserwacją jest fakt, że jeśli istnieje wierzchołek, po wybraniu którego powstaje parzysta liczba izomorficznych grafów, gracz znajdujący się w tej sytuacji również może być pewny zwycięstwa, korzystając z teorii wygrywającej gier symetrycznych. Niestety, nie udało nam się odkryć uogólnienia tej zasady na pozostałe sytuacje. 

Wybierając heurestykę dla postawionego problemu postanowiliśmy dążyć do postawienia gracza w sytuacji wygrywającej, znanej z gier symetrycznych. W tym celu stworzyliśmy funkcję heurestyczną, która każdemu wierzchołkowi będzie przypisywała wartość, tym większą, im bardziej niepodobny dany wierzchołek jest do pozostałych w grafie. Podobnieństwo obliczamy w następujący sposób: dla każdego wierzchołka obliczamy trzy zmienne: 
\begin{itemize}
\item $x_a$ - największa odległość od tego wierzchołka do dowolnego innego w tym grafie.
\item $x_b$ - liczba wierzchołków sąsiadujących z tym wierzchołkiem
\item $x_c$ - liczba wierzchołków znajdujących się w odległości 2 od danego wierzchołka. 
\end{itemize}

Z tych trzech współczynników wartość heurestyki, przy użyciu arbitralnych wag wa, wb, wc, wyznacza się jako odległość do najbliższego wierzchołka w przestrzeni tworzonej przez ten wektor odleglości. Wierzchołkiem do usunięcia przez komputerowego gracza jest wierzchołek, którego wartość heurestyki jest największa, czyli jego najbliższy sąsiad jest najdalej ze wszystkich rozpatrywanych wierzchołków.

Oto wzór na odległość między punktami x i y:

$$d(x_a,x_b,x_c,y_a,y_b,y_c)=(x_a-y_a)^2+(x_b-y_b)^2+(x_c-y_c)^2 $$

Proponowana heurestyka będzie starała się promować wierzchołki, które są zupełnie inne reszty.

\section{Struktury danych}

Implementacja algorytmu opiera się na klasie Graph, która jest reprezentacją grafu w postaci list sąsiedztwa:
\begin{lstlisting}
public class Graph {
    private Map<String, Set<String>> neighbourList;
/* konstruktor, metody */
}
\end{lstlisting}

Jak widać, informacje przechowywane są w formie mapy wierchołek - zbiór sąsiadów. Każdy wierzchołek jest opisany ciągiem znaków, najczęściej jest to jeden znak alfanumeryczny. Nasza implementacja grafu umożliwia następujące operacje:
\begin{itemize}
\item dodanie wierzchołka  - public void addVertex(String vertex)
\item dodanie krawędzi - public void addEdge(String vertexFrom, String vertexTo)
\item usunięcie wierchołka zgodnie z regułami gry - public void kaylesRemove(String vertex)
\item odczytanie zbioru wierchołków - public Set getVertices()
\item odczytanie sąsiadów wierzchołka - public Set getNeighbours(String vertex)
\end{itemize}

\section{Projekty testów}
Testy heurystyki będą przeprowadzane w trakcie gry przeciwko graczowi ludzkiemu. Do tego celu użyjemy 3 predefiniowanych grafów o rosnącym stopniu komplikacji - coraz większej liczbie wierzchołków. Na pierwszym grafie będzie można łatwo zauważyć jakie ruchy zagwarantują wygraną, na drugim grafie będzie to trudniejsze, a na ostatnim - znacznie trudniejsze. Ostatni graf będzie zawierał kilkadziesiąt wierzchołków.
Drugim etapem testów (jako że nasza heurystyka umożliwia parametryzację) jest uruchomienie gier komputer - komputer. Będą odbywały się one na znacznie większych losowych grafach (kilkaset, kilka tysięcy wierzchołków) i pokażą czy istnieją wartości parametrów, które dają lepsze wyniki.

\section{Złożoność}
Złożoność algorytmu jest zależna od jej składowych. Pierwsza z nich, czyli liczba sąsiadów wierzchołka jest w pesymistycznym wypadku liniowa. Druga z nich, liczba wierzchołków w odległości 2 od danego wierzchołka, może być policzona za pomocą algorytmu przeszukiwania wszerz, stąd jego złożoność jest w najgorszym wypadku liniowa. Trzecią składową liczy się w ten sam sposób, więc złożoność również jest liniowa. Wartości te należy policzyć dla każdego wierzchołka grafu, za każdym ruchem gracza.

Stąd, maksymalna liczba przejść zbioru wierzchołków to:

$$N*(N + N + N)$$

Co oznacza kwadratową złożoność obliczeniową. 

\section{Wejście/wyjście, warunki stopu}
Z racji, że algorytm ten będzie służył do wyboru ruchów grze, wejście będzie bezpośrednio powiązane z początkowym stanem gry. Algorytm będzie akceptował dwa rodzaje wejść. Pierwszym będzie graf, podany za pomocą liczby wierzchołków i listy krawędzi pomiędzy tymi wierzchołkami. Drugim sposobem jest podanie liczby wierzchołków i krawędzi. Algorytm wtedy sam stworzy graf, losowo rozmieszczając krawędzie między wierzchołkami. 

Algorytm nie będzie miał skomplikowanego wyjścia. W przypadku gry między dwoma komputerami będzie po prostu zwracał informację, który z nich wygrał. 

Zakończenie gry odbywa się, gry któryś z graczy nie może już wykonać ruchu. W takiej sytuacji wygrywa jego przeciwnik.


\end{document}